\section{Изменения}

\subsection{Первого приоритета}

\subsubsection{Выполнение SQL запросов}

При сохранении результатов в XML файле преобразовывать специальные символы XML, такие как
\textbf{``\&''}, \textbf{``<''}, \textbf{``>''} в корректное представление для XML формата. Сейчас
может формироваться некорректный XML файл если результаты запросов содержать специальные управляющие
символы XML.

\subsubsection{Утилита сборки дистрибутива}

Возможность сборки приложения как в 32-х, так и для 64-х битной версий Windows. Сейчас доступна
только 32-х битная сборка.

\subsubsection{Конфигурация}

Список языков для интерфейса и отчётов (два отдельных списка) сделать в системном конфигурационном
файле.

\bigskip

Выбор языка интерфейса и языка отчёта по умолчанию (для новых пользователей) при установке
программы.

\subsubsection{Добавление новой лицензии}

В меню ``Настройки'' добавить новый диалог для добавления новой лицензии. Пользователь должен
выбрать файл, который будет скопирован в домашнюю папку пользователя под именем
MSSQL\-Server\-Auditor.Connec\-tions.YYYMMDD.xml, что позволит ему начать работу с новым экземпляром
сервера. Возможность удаления одной или нескольких лицензий пользователем. Проверка валидности
лицензий.

\subsubsection{Интерфейс}

Сохранять позицию и размер главного окна программы. При остановке и рестарте приложения
восстанавливать позицию и размер окна на момент последней остановки программы.

\bigskip

Для запросов, которые останавливаются по истечении времени кнопка остановки запроса (красного цвета)
остаётся активной и контекстное меню обновления становится неактивным (невозможно выполнить другие
запросы). Сделать кнопку остановки запроса неактивной и активировать контекстное меню после
окончания выполнения запросов в том числе и когда запросы окончились в связи с превышением лимита
времени или закончились неудачно.

\bigskip

Обновление данных по нажатию \textbf{F5} при просмотре отчёта.

\bigskip

Контекстное меню, ``Открыть вложенные узлы'' или ``Закрыть вложенные узлы'' в дополнение к
контекстному меню ``Обновить узел'', ``Обновить вложенные узлы''.

\bigskip

Для отображаемого дерева слева на главном окне ввести возможность отображения иконки из XML файла,
где определяются HTML, XML, DataGrid элементы. В утилиту формирования дистрибутива добавить
возможность введения иконок как доступные ресурсы приложения. Пример отображения элементов с
иконками посмотреть в Micro\-soft SQL Mana\-gement Studio при выводе списка баз данных.

\subsubsection{Шрифты в диалоговых окнах}

Использовать системные настройки, не использовать никакие шрифты, особенно шрифты указанные в
исходных текстах. В случае необходимости подготавливать файлы ресурсов.

\begin{lstlisting}[language=C,label=Fonts:frmErrorLog.Designer.cs,caption=frmErrorLog.Designer.cs]

	99 this.lbInstance.Font = new System.Drawing.Font("Tahoma", ...);

\end{lstlisting}

\subsubsection{Отключение меню}

Сделать возможным отключение пользовательского меню программы. Идея в том, чтобы пользователь,
работающий на мониторе с низким разрешением, мог сосредоточился на отчётах и лишние интерфейсные
меню и строки можно было бы отключить. Сделать возможным включение меню через изменение системного
меню. Т.е. изменить системное меню Windows добавив пункт включить пользовательское меню.

\subsection{Второго приоритета}

При сохранении настроек показа таблиц в файле ``UserSettings'' в качестве идентификатора объекта
учитывать тип процессора (сейчас только ``DataGridPreprocessorDialog'', но может добавиться в
будущем), язык отображения и название шаблона. Так как могут быть одинаковые таблицы (один и тот же
``id'') в разных шаблонах.

Вынести настройки показа таблиц в другой файл. Сейчас используется файл ``UserSettings'', сделать
файл ``UserLayoutSettings''. Причина в том что в файле ``UserSettings'' хранятся настройки профиля,
которые определяет пользователь сам, в файле ``UserLayoutSettings'' автоматические, будет разумно их
разделить.

\subsubsection{Отображение графиков}

Для графиков вида ``NameDateStackedColumn'' при наличии только одной даты всё равно отображается
начальная дата ``30/12/1899'', хотя в данных такой даты нет. Убрать показ дат, которых нет в данных.
При наличии только одной даты показывать только один день.

\subsection{Третьего приоритета}

\subsubsection{Интерфейс -- вкладки}

Вкладки для подключений сделать размещающимися только над деревом списка слева, а вкладки для
графиков и текста поднять на один уровень выше, что позволит графикам и таблицам иметь больше места
на экране. Сейчас в случае одного подключения место не используется.

\subsubsection{Интерфейс -- прогресс бар}

Закрашивать отображение имени соединения в панели статуса, вместо выделения отдельной полосы слева.
Также обратить внимание что сейчас область прогресс бара отличается от других элементов в строке
статуса. Перенести кнопку остановки выполнения запроса вверх на уровень вкладки.

\subsubsection{Интерфейс -- вкладки}

Опция в диалоге конфигурации - при только одной открытой вкладке не показывать её или показывать.
При нескольких вкладках - показывать открытые вкладки всегда. Поведение похоже на браузеры
интернета, там есть такие настройки.

\subsubsection{Интерфейс -- строка обновления}

Если не было обновления страницы - показывать символы ``--'' в строке даты, чтобы статусная строка
не была плавающей. Дата всегда показывалась на одном и том же месте.

\subsubsection{Вывод графиков -- параметры}

Возможность указания единиц измерения осей графика. Параметр графика с названием оси и указанием
единиц измерения.

\bigskip

При обновлении графика не сохраняются скрытые элементы. Сохранять конфигурация скрытия / отображения
элементов при обновлении узла.

\subsubsection{Настройки программы}

При просмотре соединения в списке процессов MS SQL Server в графе ``ProgramName'' утилита должна
прописывать себя как \textbf{MSSQLServerAuditor} \textbf{0.5.0.1}, где \textbf{0.5.0.1} номер
текущей версии утититы. Для этого изменить строку подключения к серверу (параметр тип приложения).

\bigskip

Сейчас временные файлы и настройки программы хранятся в папке ``Документы'', в моём случае это папка
``C:{\textbackslash}Docu\-ments and
Set\-tings{\textbackslash}a\_sayche\-nkoa{\textbackslash}Docu\-ments''. Думаю что более логично
поместить файлы в папку ``Приложения''. В моём случае это будет папка
``C:{\textbackslash}Docu\-ments and
Set\-tings{\textbackslash}a\_sayche\-nkoa{\textbackslash}Appli\-cation Data''. Почему? Папка
``Документы'' предназначена для документов, которые пользователь создаёт сам, а в данном случае все
файлы генерируются приложением автоматически.

\subsubsection{Конфигурационные файлы}

Изменить имя файла с локализацией интерфейса программы. Сейчас файл называется
``MSSQL\-Server\-Auditor.\-MSSQLQuery.\-i18n.\-xml''. Предлагаю изменить на
``MSSQL\-Server\-Auditor.\-i18n.\-xml''. Файл не имеет отношения к файлу с запросами как может
показаться по его имени.

\subsubsection{Проверка версии .Net при старте}

В настоящее время при старте приложения проверяется версия .Net и при несоответствии версии,
выдаётся диалоговое сообщение с требованием установки обновления .Net. Нужно локализовать сообщение
на нужном языке и показывать текущую установленную версию .Net а также версии и разрядности (32, 64
бита) текущей операционной системы.

\section{Пояснения и комментарии}

