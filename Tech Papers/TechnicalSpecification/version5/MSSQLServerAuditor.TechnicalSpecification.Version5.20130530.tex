\documentclass[10pt,a4paper]{article}
\usepackage[utf8x]{inputenc}
\usepackage[left=2.5cm,right=2.0cm,top=1.75cm,bottom=1.75cm,bindingoffset=0cm]{geometry}
\usepackage[english,russian]{babel}
\usepackage[usenames,dvipsnames]{color}
\usepackage{caption}
\usepackage{listings}
\usepackage{multirow}
\usepackage{tabularx}
\usepackage{titlesec}
%\usepackage{hyperref}
\usepackage{lastpage}

\definecolor{gray}{rgb}{0.4,0.4,0.4}
\definecolor{darkblue}{rgb}{0.0,0.0,0.6}
\definecolor{cyan}{rgb}{0.0,0.6,0.6}

\setcounter{secnumdepth}{4}

\setlength{\parindent}{0cm}

\titleformat{\paragraph}
{\normalfont\normalsize\bfseries}{\theparagraph}{1em}{}
\titlespacing*{\paragraph}
{0pt}{3.25ex plus 1ex minus .2ex}{1.5ex plus .2ex}

% this style should be active for all lstlistings environments
\lstdefinestyle{all}{
	basicstyle       = \small\sffamily,
	captionpos       = t,
	columns          = fullflexible,
	extendedchars    = false,
	frame            = lrtb,
	framerule        = 1pt,
	framexleftmargin = 1pt,
	inputencoding    = utf8x,
	keepspaces       = true,
	showstringspaces = false,
	tabsize          = 2
}

% this style should be active (additionally to "all") for "input" lstlistings environments
\lstdefinestyle{input}{
	commentstyle     = \itshape\color{red},
	tabsize          = 2
}

\lstset{
	basicstyle       = \small\sffamily,
	captionpos       = t,
	columns          = fullflexible,
	commentstyle     = \color{gray}\upshape,
	extendedchars    = false,
	frame            = 0rtb,
	framerule        = 0pt,
	framexleftmargin = 0pt,
	inputencoding    = utf8x,
	keepspaces       = true,
	language         = C,
	showstringspaces = false,
	stringstyle      = \bf,
	tabsize          = 2
}

\lstset{
	basicstyle       = \small\sffamily,
	captionpos       = t,
	columns          = fullflexible,
	commentstyle     = \color{gray}\upshape,
	extendedchars    = false,
	frame            = 0rtb,
	framerule        = 0pt,
	framexleftmargin = 0pt,
	inputencoding    = utf8x,
	keepspaces       = true,
	language         = XML,
	showstringspaces = false,
	stringstyle      = \bf,
	tabsize          = 2
}

%\lstset{ %
%	language=XML,                  % выбор языка для подсветки (здесь это XML)
%	%basicstyle=\small\sffamily,    % размер и начертание шрифта для подсветки кода
%	%numbers=left,                  % где поставить нумерацию строк (слева\справа)
%	%numberstyle=\tiny,             % размер шрифта для номеров строк
%	%stepnumber=1,                  % размер шага между двумя номерами строк
%	%numbersep=5pt,                 % как далеко отстоят номера строк от подсвечиваемого кода
%	backgroundcolor=\color{white}, % цвет фона подсветки - используем \usepackage{color}
%	showspaces=false,              % показывать или нет пробелы специальными отступами
%	showstringspaces=false,        % показывать или нет пробелы в строках
%	showtabs=false,                % показывать или нет табуляцию в строках
%	%frame=single,                  % рисовать рамку вокруг кода
%	tabsize=2,                     % размер табуляции по умолчанию равен 2 пробелам
%	%captionpos=t,                  % позиция заголовка вверху [t] или внизу [b]
%	breaklines=true,               % автоматически переносить строки (да\нет)
%	breakatwhitespace=false,       % переносить строки только если есть пробел
%	%escapeinside={\%*}{*)},        % если нужно добавить комментарии в коде
%	morekeywords={encoding,xs:schema,xs:element,xs:complexType,xs:sequence,xs:attribute}
%}

\lstdefinelanguage{XML}{
	%basicstyle      = \ttfamily\color{darkblue}\bfseries,
	basicstyle      = \ttfamily,
	morestring      = [b]",
	morestring      = [s]{>}{<},
	morecomment     = [s]{<?}{?>},
	stringstyle     = \color{black},
	identifierstyle = \color{darkblue},
	keywordstyle    = \color{cyan},
	morekeywords    = {xmlns,version,type}% list your attributes here
}

\DeclareCaptionFont{white}{\color{white}}
\DeclareCaptionFormat{listing}{\colorbox[cmyk]{0.43, 0.35, 0.35,0.01}{\parbox{\textwidth}{\hspace{15pt}#1#2#3}}}
\captionsetup[lstlisting]{format=listing,labelfont=white,textfont=white, singlelinecheck=false, margin=0pt, font={bf,footnotesize}}
\renewcommand{\lstlistingname}{Пример}

\let\oldquote\quote
\renewcommand\quote{\small\oldquote}

\setcounter{tocdepth}{4}

\begin{document}

\begin{titlepage}

\begin{center}

	\Large{Утверждено}

\end{center}

\hrulefill

\vspace{8em}

\begin{center}

	\Large{\textbf{Техническое задание}}

\end{center}

\vspace{6em}

\begin{center}

	\Large{Доработка утилиты генерации отчёта}

	\vspace{1.5em}

	\Large{сервера баз данных MS SQL Server{\copyright}}

	\vspace{2.5em}

	\Large{версия 0.5.0.1}

\end{center}

\vspace{6em}

\vspace{\fill}

\begin{center}

	\Large{Листов \pageref{LastPage}}

\end{center}

\vspace{6em}

\hrulefill

\begin{center}

	\Large{\today}

	\vspace{1.5em}

	\Large{Сидней, Австралия, 2013}

\end{center}

\end{titlepage}



\tableofcontents

\section{Общие сведения}

\subsection{Названия}

Полное наименование системы: Система построения отчетов о системе управления базами данных MS SQL
Server{\copyright} MSSQLServerAuditor. Краткое наименование системы: MSSQLServerAuditor, программа,
утилита, система.

\subsection{Назначение системы}

Обновление программы (новая версия 0.5.0.1) генерации отчёта об экземпляре базы данных и параметрах
работы сервера базы данных, определяемых конфигурационным файлом на основе требований Заказчика.

\subsection{Требования}

\subsubsection{Требования к информационной и программной совместимости}

Модель системы должна работать под управлением ОС Windows 7, поэтому требуется совместимость
исполняемого модуля и библиотек динамического подключения стандартам, используемым этими ОС на
платформе IBM PC. Для доступа к базам данных требуется наличие установленного ядра работы с БД
Microsoft SQL Server (клиенская часть) и библиотеки .Net 4.x.

\subsubsection{Требования к процессу разработки программы}

Исполнитель руководствуется соглашением по оформлению кода команды RSDN
(http://www.rsdn.ru/\-article\-/mag/\-200401/\-codestyle.\-XML). В случае неопределённостей или
неясностей Заказчик и Исполнитель вырабатывают дополнение к стандарту RSDN (не противоречащие ему)
до начала разработки. При приемке каждого этапа Заказчик вправе настаивать на оформлении кода
программной разработки в соответствии со стилем RSDN до закрытия соответствующего этапа.

\subsubsection{Требования к программной документации}

Исполнитель подготавливает комментарии в программе в формате необходимом для автоматической
генерации документации средствами выбранной системы разработки. Исполнитель подготавливает
инструкции по генерации документации и необходимые программные утилиты. Документация должна быть
полностью автоматически сгенерирована и обновляется в конце каждого этапа. Наличие обновленной
документации обязательное требование приемки этапа.

\subsection{Соглашения}

\subsubsection{Права собственности}

Заказчик обладает полным правом собственности на весь разработанный программный код, документацию и
прочие документы. Исполнитель обязан удалить весь разработанный программный код и сопровождающие
документы после окончания проекта. Упоминание участия Исполнителя в разработке возможно только при
получении согласия Заказчика в письменном виде.

\subsubsection{Порядок разрешения споров}

Стороны договариваются о разрешении любых споров во внесудебном порядке путем проведения
переговоров. При несогласии между Заказчиком и Исполнителем после переговоров дальнейшая разработка
и сотрудничество программы прекращается и оплата за последний этап не производится. Заказчик
настаивает на безусловном соблюдении всех пунктов данного технического задания и дополнительных
документов к техническому заданию Исполнителем.

\section{Разработка}

\subsection{Основания для разработки}

Проект Заказчика. Договор между Исполнителем и Заказчиком на основании проведения конкурса
Free\-Lance.ru

\subsection{Назначение разработки}

Коммерческий проект Заказчика.

\subsection{Источники и порядок финансирования}

Собственные средства Заказчика.

\subsection{Порядок оформления и предъявления заказчику результатов работ}

Заказчик оценивает работу и информирует Исполнителя путем заведения тикетов об ошибках и
несоответствиях. Исполнитель выкладывает все программные коды, документацию, технические
руководства, другие сопровождающие документы на выбранный сервер управления хранением исходных кодов
программ.

\subsection{Порядок контроля и приемки}

Оплата каждого из завершенных этапов производится после тестирования Заказчиком и подписания актов
приемки на основе итоговой таблицы приёма работы (акт составляет Исполнитель). Каждый этап оплаты
фиксируется в дополнительном документе, приложению к техническому заданию.

\subsection{Итоговая таблица прима работы}

\begin{table}
\begin{center}
\noindent\begin{tabularx}{\linewidth}{|c|X|c|}

		\hline

		\textbf{N} & \textbf{Тип проверки} & \textbf{Стоимость} \\

		\hline

\end{tabularx}
\end{center}
\end{table}




\section{Изменения}

\subsection{Первого приоритета}

\subsubsection{Выполнение SQL запросов}

При сохранении результатов в XML файле преобразовывать специальные символы XML, такие как
\textbf{``\&''}, \textbf{``<''}, \textbf{``>''} в корректное представление для XML формата. Сейчас
может формироваться некорректный XML файл если результаты запросов содержать специальные управляющие
символы XML.

\subsubsection{Утилита сборки дистрибутива}

Возможность сборки приложения как в 32-х, так и для 64-х битной версий Windows. Сейчас доступна
только 32-х битная сборка.

\subsubsection{Конфигурация}

Список языков для интерфейса и отчётов (два отдельных списка) сделать в системном конфигурационном
файле.

\bigskip

Выбор языка интерфейса и языка отчёта по умолчанию (для новых пользователей) при установке
программы.

\subsubsection{Добавление новой лицензии}

В меню ``Настройки'' добавить новый диалог для добавления новой лицензии. Пользователь должен
выбрать файл, который будет скопирован в домашнюю папку пользователя под именем
MSSQL\-Server\-Auditor.Connec\-tions.YYYMMDD.xml, что позволит ему начать работу с новым экземпляром
сервера. Возможность удаления одной или нескольких лицензий пользователем. Проверка валидности
лицензий.

\subsubsection{Интерфейс}

Сохранять позицию и размер главного окна программы. При остановке и рестарте приложения
восстанавливать позицию и размер окна на момент последней остановки программы.

\bigskip

Для запросов, которые останавливаются по истечении времени кнопка остановки запроса (красного цвета)
остаётся активной и контекстное меню обновления становится неактивным (невозможно выполнить другие
запросы). Сделать кнопку остановки запроса неактивной и активировать контекстное меню после
окончания выполнения запросов в том числе и когда запросы окончились в связи с превышением лимита
времени или закончились неудачно.

\bigskip

Обновление данных по нажатию \textbf{F5} при просмотре отчёта.

\bigskip

Контекстное меню, ``Открыть вложенные узлы'' или ``Закрыть вложенные узлы'' в дополнение к
контекстному меню ``Обновить узел'', ``Обновить вложенные узлы''.

\bigskip

Для отображаемого дерева слева на главном окне ввести возможность отображения иконки из XML файла,
где определяются HTML, XML, DataGrid элементы. В утилиту формирования дистрибутива добавить
возможность введения иконок как доступные ресурсы приложения. Пример отображения элементов с
иконками посмотреть в Micro\-soft SQL Mana\-gement Studio при выводе списка баз данных.

\subsubsection{Шрифты в диалоговых окнах}

Использовать системные настройки, не использовать никакие шрифты, особенно шрифты указанные в
исходных текстах. В случае необходимости подготавливать файлы ресурсов.

\begin{lstlisting}[language=C,label=Fonts:frmErrorLog.Designer.cs,caption=frmErrorLog.Designer.cs]

	99 this.lbInstance.Font = new System.Drawing.Font("Tahoma", ...);

\end{lstlisting}

\subsubsection{Отключение меню}

Сделать возможным отключение пользовательского меню программы. Идея в том, чтобы пользователь,
работающий на мониторе с низким разрешением, мог сосредоточился на отчётах и лишние интерфейсные
меню и строки можно было бы отключить. Сделать возможным включение меню через изменение системного
меню. Т.е. изменить системное меню Windows добавив пункт включить пользовательское меню.

\subsection{Второго приоритета}

При сохранении настроек показа таблиц в файле ``UserSettings'' в качестве идентификатора объекта
учитывать тип процессора (сейчас только ``DataGridPreprocessorDialog'', но может добавиться в
будущем), язык отображения и название шаблона. Так как могут быть одинаковые таблицы (один и тот же
``id'') в разных шаблонах.

Вынести настройки показа таблиц в другой файл. Сейчас используется файл ``UserSettings'', сделать
файл ``UserLayoutSettings''. Причина в том что в файле ``UserSettings'' хранятся настройки профиля,
которые определяет пользователь сам, в файле ``UserLayoutSettings'' автоматические, будет разумно их
разделить.

\subsubsection{Отображение графиков}

Для графиков вида ``NameDateStackedColumn'' при наличии только одной даты всё равно отображается
начальная дата ``30/12/1899'', хотя в данных такой даты нет. Убрать показ дат, которых нет в данных.
При наличии только одной даты показывать только один день.

\subsection{Третьего приоритета}

\subsubsection{Интерфейс -- вкладки}

Вкладки для подключений сделать размещающимися только над деревом списка слева, а вкладки для
графиков и текста поднять на один уровень выше, что позволит графикам и таблицам иметь больше места
на экране. Сейчас в случае одного подключения место не используется.

\subsubsection{Интерфейс -- прогресс бар}

Закрашивать отображение имени соединения в панели статуса, вместо выделения отдельной полосы слева.
Также обратить внимание что сейчас область прогресс бара отличается от других элементов в строке
статуса. Перенести кнопку остановки выполнения запроса вверх на уровень вкладки.

\subsubsection{Интерфейс -- вкладки}

Опция в диалоге конфигурации - при только одной открытой вкладке не показывать её или показывать.
При нескольких вкладках - показывать открытые вкладки всегда. Поведение похоже на браузеры
интернета, там есть такие настройки.

\subsubsection{Интерфейс -- строка обновления}

Если не было обновления страницы - показывать символы ``--'' в строке даты, чтобы статусная строка
не была плавающей. Дата всегда показывалась на одном и том же месте.

\subsubsection{Вывод графиков -- параметры}

Возможность указания единиц измерения осей графика. Параметр графика с названием оси и указанием
единиц измерения.

\bigskip

При обновлении графика не сохраняются скрытые элементы. Сохранять конфигурация скрытия / отображения
элементов при обновлении узла.

\subsubsection{Настройки программы}

При просмотре соединения в списке процессов MS SQL Server в графе ``ProgramName'' утилита должна
прописывать себя как \textbf{MSSQLServerAuditor} \textbf{0.5.0.1}, где \textbf{0.5.0.1} номер
текущей версии утититы. Для этого изменить строку подключения к серверу (параметр тип приложения).

\bigskip

Сейчас временные файлы и настройки программы хранятся в папке ``Документы'', в моём случае это папка
``C:{\textbackslash}Docu\-ments and
Set\-tings{\textbackslash}a\_sayche\-nkoa{\textbackslash}Docu\-ments''. Думаю что более логично
поместить файлы в папку ``Приложения''. В моём случае это будет папка
``C:{\textbackslash}Docu\-ments and
Set\-tings{\textbackslash}a\_sayche\-nkoa{\textbackslash}Appli\-cation Data''. Почему? Папка
``Документы'' предназначена для документов, которые пользователь создаёт сам, а в данном случае все
файлы генерируются приложением автоматически.

\subsubsection{Конфигурационные файлы}

Изменить имя файла с локализацией интерфейса программы. Сейчас файл называется
``MSSQL\-Server\-Auditor.\-MSSQLQuery.\-i18n.\-xml''. Предлагаю изменить на
``MSSQL\-Server\-Auditor.\-i18n.\-xml''. Файл не имеет отношения к файлу с запросами как может
показаться по его имени.

\subsubsection{Проверка версии .Net при старте}

В настоящее время при старте приложения проверяется версия .Net и при несоответствии версии,
выдаётся диалоговое сообщение с требованием установки обновления .Net. Нужно локализовать сообщение
на нужном языке и показывать текущую установленную версию .Net а также версии и разрядности (32, 64
бита) текущей операционной системы.

\section{Пояснения и комментарии}



\end{document}

