\section{Общие сведения}

\subsection{Названия}

Полное наименование системы: Система построения отчетов о системе управления базами данных MS SQL
Server{\copyright} MSSQLServerAuditor. Краткое наименование системы: MSSQLServerAuditor, программа,
утилита, система.

\subsection{Назначение системы}

Обновление программы (новая версия 0.5.0.1) генерации отчёта об экземпляре базы данных и параметрах
работы сервера базы данных, определяемых конфигурационным файлом на основе требований Заказчика.

\subsection{Требования}

\subsubsection{Требования к информационной и программной совместимости}

Модель системы должна работать под управлением ОС Windows 7, поэтому требуется совместимость
исполняемого модуля и библиотек динамического подключения стандартам, используемым этими ОС на
платформе IBM PC. Для доступа к базам данных требуется наличие установленного ядра работы с БД
Microsoft SQL Server (клиенская часть) и библиотеки .Net 4.x.

\subsubsection{Требования к процессу разработки программы}

Исполнитель руководствуется соглашением по оформлению кода команды RSDN
(http://www.rsdn.ru/\-article\-/mag/\-200401/\-codestyle.\-XML). В случае неопределённостей или
неясностей Заказчик и Исполнитель вырабатывают дополнение к стандарту RSDN (не противоречащие ему)
до начала разработки. При приемке каждого этапа Заказчик вправе настаивать на оформлении кода
программной разработки в соответствии со стилем RSDN до закрытия соответствующего этапа.

\subsubsection{Требования к программной документации}

Исполнитель подготавливает комментарии в программе в формате необходимом для автоматической
генерации документации средствами выбранной системы разработки. Исполнитель подготавливает
инструкции по генерации документации и необходимые программные утилиты. Документация должна быть
полностью автоматически сгенерирована и обновляется в конце каждого этапа. Наличие обновленной
документации обязательное требование приемки этапа.

\subsection{Соглашения}

\subsubsection{Права собственности}

Заказчик обладает полным правом собственности на весь разработанный программный код, документацию и
прочие документы. Исполнитель обязан удалить весь разработанный программный код и сопровождающие
документы после окончания проекта. Упоминание участия Исполнителя в разработке возможно только при
получении согласия Заказчика в письменном виде.

\subsubsection{Порядок разрешения споров}

Стороны договариваются о разрешении любых споров во внесудебном порядке путем проведения
переговоров. При несогласии между Заказчиком и Исполнителем после переговоров дальнейшая разработка
и сотрудничество программы прекращается и оплата за последний этап не производится. Заказчик
настаивает на безусловном соблюдении всех пунктов данного технического задания и дополнительных
документов к техническому заданию Исполнителем.

\section{Разработка}

\subsection{Основания для разработки}

Проект Заказчика. Договор между Исполнителем и Заказчиком на основании проведения конкурса
Free\-Lance.ru

\subsection{Назначение разработки}

Коммерческий проект Заказчика.

\subsection{Источники и порядок финансирования}

Собственные средства Заказчика.

\subsection{Порядок оформления и предъявления заказчику результатов работ}

Заказчик оценивает работу и информирует Исполнителя путем заведения тикетов об ошибках и
несоответствиях. Исполнитель выкладывает все программные коды, документацию, технические
руководства, другие сопровождающие документы на выбранный сервер управления хранением исходных кодов
программ.

\subsection{Порядок контроля и приемки}

Оплата каждого из завершенных этапов производится после тестирования Заказчиком и подписания актов
приемки на основе итоговой таблицы приёма работы (акт составляет Исполнитель). Каждый этап оплаты
фиксируется в дополнительном документе, приложению к техническому заданию.

\subsection{Итоговая таблица прима работы}

\begin{table}
\begin{center}
\noindent\begin{tabularx}{\linewidth}{|c|X|c|}

		\hline

		\textbf{N} & \textbf{Тип проверки} & \textbf{Стоимость} \\

		\hline

\end{tabularx}
\end{center}
\end{table}


