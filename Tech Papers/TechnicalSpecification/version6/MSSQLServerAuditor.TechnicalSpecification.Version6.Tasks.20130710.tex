\section{Изменения}

\subsection{Первого приоритета}

\subsubsection{Отображение дерева запросов}

При выполнении запросов в настоящее время периодически перерисовывается дерево, возникает эффект
``моргания'', очень неприятно. Попытаться минимизировать перерисовку графических компонентов во
время выполнения запросов, по возможности не перерисовывать ничего во время выполнения запросов, а
только после окончания выполения.

\subsubsection{Интерфейс главного дерева}

В дополение к интерфейсу с табами добавить интерфейс где все компоненты происходят от одного корня с
названием типа подключения. Сделать дополнительный интерфейс в стиле SQL Server Management Studio,
как там сделано для подключений (есть единое дерево со всеми подключениями). Добавить новую опцию
настроек в меню конфигурации, настройка идёт на уровне пользователя. Сохранить существующий
интерфейс, дополнительный предоставлять в качестве опции.

\subsubsection{Отображение дерева запросов}

Показывать общее число записей для дерева запросов слева. Для этого суммировать все возвращённые
результаты и выводить общее число записей. Отображать число в круглых скобках после названия меню.
Обратить внимание на отображение числа для запросов вида scope="database" где для каждой базы данных
отображать число относящееся к ней только, на верхнем уровне суммировать для всех баз данных.

\subsubsection{Отображение дерева запросов (не отображать пустые записи)}

Добавить новую опцию настройки которая позволяет для запросов вида scope="database" не отображать те
базы данных для которых число записей равно нулю. Добавить опцию настройки для отображения или не
отображения числа возвращённых записей в главном дереве запросов.

\subsubsection{Вывод для запросов к нескольким базам}

Для запросов вида scope="database" есть XML файл для вывода данных каждой из баз, но нет возможности
задать файл вывода для корневой папки. Например мы формируем список объектов для каждой из баз
данных, а в папке верхнего уровня хотим вывести справочную информацию что в нижеследующих папках
находятся списки объектов.

\subsubsection{Формирование XML результата для запросов к нескольким базам}

При формировании результирующего XML файла для запросов вида scope="database" некорретно формируется
MSSQLResult. В файл для одной базы данных попадают результаты для всех остальных баз данных.

\begin{lstlisting}[language=XML,label=current_result_set,caption=Выводится сейчас]

	<?xml version="1.0" encoding="UTF-8"?>
	<MSSQLResults timestamp="2013-07-12T11:21:01" database="FCM15">
	  <MSSQLResult name="Get" hierarchy="" RowCount="0" SqlErrorCode="" SqlErrorNumber="" />
	  <MSSQLResult name="Get" hierarchy="" RowCount="0" SqlErrorCode="" SqlErrorNumber="" />
	  <MSSQLResult name="Get" hierarchy="" RowCount="150" SqlErrorCode="" SqlErrorNumber="" />
	  <MSSQLResult name="Get" hierarchy="" RowCount="8" SqlErrorCode="" SqlErrorNumber="" />
	  <MSSQLResult name="Get" hierarchy="" RowCount="3" SqlErrorCode="" SqlErrorNumber="" />
	  <MSSQLResult name="Get" hierarchy="" RowCount="11" SqlErrorCode="" SqlErrorNumber="" />
	  <MSSQLResult name="Get" hierarchy="" RowCount="4" SqlErrorCode="" SqlErrorNumber="" />
	  <MSSQLResult name="Get" hierarchy="" RowCount="0" SqlErrorCode="" SqlErrorNumber="" />
	  <MSSQLResult name="Get" hierarchy="" RowCount="0" SqlErrorCode="" SqlErrorNumber="" />
	  <MSSQLResult name="Get" hierarchy="" RowCount="0" SqlErrorCode="" SqlErrorNumber="" />
	  <MSSQLResult name="Get" hierarchy="" RowCount="20" SqlErrorCode="" SqlErrorNumber="" />
	  <MSSQLResult name="Get" hierarchy="" RowCount="0" SqlErrorCode="" SqlErrorNumber="" />
	  <MSSQLResult name="Get" hierarchy="" RowCount="10" SqlErrorCode="" SqlErrorNumber="" />
	  <MSSQLResult name="Get" hierarchy="" RowCount="91" SqlErrorCode="" SqlErrorNumber="" />
	  <MSSQLResult name="Get" hierarchy="" RowCount="69" SqlErrorCode="" SqlErrorNumber="" />
	  <MSSQLResult name="Get" hierarchy="" RowCount="0" SqlErrorCode="" SqlErrorNumber="" />
	  <MSSQLResult name="Get" hierarchy="" RowCount="0" SqlErrorCode="" SqlErrorNumber="" />
	  <MSSQLResult name="Get" hierarchy="" RowCount="2" SqlErrorCode="" SqlErrorNumber="" />
	  <MSSQLResult name="Get" hierarchy="" RowCount="0" SqlErrorCode="" SqlErrorNumber="" />
	  <MSSQLResult name="Get" hierarchy="" RowCount="0" SqlErrorCode="" SqlErrorNumber="" />
	  <Get>
	     ...
	  </Get>
	</MSSQLResults>

\end{lstlisting}

\begin{lstlisting}[language=XML,label=new_result_set,caption=Нужно выводить]

	<?xml version="1.0" encoding="UTF-8"?>
	<MSSQLResults timestamp="2013-07-12T11:21:01" database="FCM15">
	  <MSSQLResult name="Get" hierarchy="" RowCount="150" SqlErrorCode="" SqlErrorNumber="" />
	  <Get>
	     ...
	  </Get>
	</MSSQLResults>

\end{lstlisting}

\subsubsection{Обновление}

Не всегда корректно работает обновление вложенных узлов на главном дереве. (правая клавиша мыши -
выбрать ``Обновить вложенные узлы'').

\subsubsection{Старт приложение}

При старте приложения не проверять доступность серверов, делать это при каждом запуске запроса так
как в момент старта приложения сервер баз данных может при некоторых обстоятельствах быть
недоступен. Если сервер не запущен или недоступен, то выводить сообщение об ошибке в тексте запроса,
но при следущем запуске снова обращаться к серверу, он может стать доступен.

\subsubsection{Выполнение SQL запросов}

При сохранении результатов в XML файле в теге \textbf{MSSQLResult} вставлять имя экземпляра сервера
баз данных как \textbf{Instance=""}, так как сейчас, в случае формирования результата для нескольких
экземпляров, неясно к какому именно экземпляру относится результат. Переменную
\textbf{SqlErrorNumber} возвращать в виде числа. При отсутствии ошибки возвращать ноль, вместо
пустой строки.

\begin{lstlisting}[language=XML,label=current_result_set,caption=Выводится сейчас]

	<MSSQLResult
		name="Get"
		hierarchy=""
		RowCount="1"
		SqlErrorCode=""
		SqlErrorNumber=""
	/>

\end{lstlisting}

\begin{lstlisting}[language=XML,label=new_result_set,caption=Нужно выводить]

	<MSSQLResult
		Instance="I"
		name="Get"
		hierarchy=""
		RowCount="1"
		SqlErrorCode=""
		SqlErrorNumber="0"
	/>

\end{lstlisting}

\subsubsection{Offline режим}

В \textbf{Offline} режиме (отчёты формируются из сохранённого файла, а не из реального сервера баз
данных) все промежуточные html файлы следует хранить не в корневой директории, а создавать отдельную
папку с названием файла в качестве имени папки и уже в ней хранить все промежуточные и временные
файлы, которые генерируются утилитой. Иначе происходит замусоривание корневой директории.

\subsubsection{Многократные запросы обновления}

В том случае когда пользователь обновляет элемент в момент уже идущего обновления ставить этот
запрос в очередь и после окончания обновления выполнять его. В случае если в очереди есть несколько
одинаковых запросов, выполнять из единственный раз.

\subsection{Второго приоритета}

\subsubsection{VARBINARY переменные}

Корректно отображать переменные типа \textbf{VARBINARY}. Сейчас переменные этого типа возвращаются
как \textbf{System.Byte}. Необходимо отображать их в виде строки вида \textbf{0x12345789}.

\subsubsection{Меню Help}

Сейчас в меню ``О программе'' отображаются версии установленных компонентов. Реорганизовать меню в
виде таблицы и показывать три колонки, в первой - название компонента, во второй установленную
версию, в третьей - битность данного компонента. Обратить особое внимание на установленную версию
\textbf{.Net runtime}. При наличии нескольких версий выводить все версии и номер версии. Добавить
вывод версии ADO Database Runtime Version (а если не установлена явно указывать что версия не
установлена). Добавить вывод версии Windows, тип версии (64 или 32 битная), номер сервис пака.
Добавить вывод установленной версии IE. Для проверки корректности работы использовать утилиту
\textbf{Free DotNet VersionCheck Utility}, расположенную по адресу
http://www.tmgdevelopment.co.uk/versioncheck.htm. Проверить правильность определения версии .Net для
версии 4.5. Ссылка
http://stackoverflow.com/\-questions/\-199080/\-how-to-detect-\-what-net-framework\--versions-and\--service-packs-are-installed

\subsubsection{Протоколирование работы}

Добавить системную переменную в конфигурацию, включение которой позволит протоколировать работу
приложения. Протоколирование нужно для решения проблем работы приложения. В настоящее время при
проблемах с запуском крайне сложно идентифицировать проблему. Все логи складывать в домашнюю
директорию пользователя ``Документы''. Логирование производить в файл с номером дня и создавать
новый файл, если такого файла ещё нет, например файл логирования
MSSQL\-Server\-Auditor.\-YYYYMMDD.\-log

\subsubsection{Строка статуса}

При длинном названии сервера (имя подключения) строка статуса с полем имя подключения становится
двойной, не помещается имя сервера, хотя есть достаточно места слева. Впечатление что ограничена
максимальная длина имени подключения.

\subsubsection{Проблемы запуска под .Net 4.0.30319.1}

Установлено, что приложение запускается, но не работает под \textbf{.Net 4.0.30319.1}. Установка
\textbf{.Net 4.5} решает проблему. Разобраться почему и в принципе если это можно то решить проблему
запуска, если требуются компоненты \textbf{.Net}, доступные только в версии \textbf{4.5}, то
объяснить какие.

\subsubsection{Конфигурация приложения}

Переименовать директорию SQLAuditor в проекте как MSSQLServerAuditor.

\subsubsection{Расположение конфигурационных файлов}

Перенести файлы \textbf{MSSQL\-Server\-Auditor\-.User\-Settings\-.xml} и
\textbf{MSSQL\-Server\-Auditor\-.UserLayout\-Settings\-.xml} в директорию ``Документы''.

\subsubsection{Конфигурация приложения (системный файл)}

Добавить переменную в системный конфигурационный файл с именем директории, где хранятся лицензии.
Директория должна находится в локальной папке пользователя, где хранятся документы.

\subsection{Третьего приоритета}

\subsubsection{Графика с данными плавающей запятой}

Не работает форматирование в графиках с данными, представленными в виде данных с плавающей запятой.
График формируется неправильно и не работает форматирование.

Вот исходные данные:

\begin{lstlisting}[language=XML,label=GanttChartData,caption=Данные для диаграммы]

	<?xml version="1.0" encoding="UTF-8"?>
	<MSSQLResults timestamp="2013-08-15T12:45:22">
	  <MSSQLResult name="GetDatabaseRestoreHistory" hierarchy="" RowCount="1" SqlErrorCode="" SqlErrorNumber="" />
	  <GetDatabaseRestoreHistory>
		   <Instance>PRODSYD041P</Instance>
		   <DatabaseName>TallymanOffline</DatabaseName>
		   <RestoreType>D</RestoreType>
		   <RestoreDate>2013-08-14T13:18:45</RestoreDate>
		   <DatabaseRestoreSizeMB>32.81</DatabaseRestoreSizeMB>
	  </GetDatabaseRestoreHistory>
	</MSSQLResults>

\end{lstlisting}

Вот декларация для диаграммы:

\begin{lstlisting}[language=XML,label=GanttChartDefinition,caption=Определение для диаграммы]

	<Format>
		#,###.##
	</Format>

	<GraphSource xsi:type="XmlFileGraphSource">
		<FileName>$INPUT$</FileName>
		<PathToItems>MSSQLResults</PathToItems>
		<ItemTag>GetDatabaseRestoreHistory</ItemTag>
		<DateTag>RestoreDate</DateTag>
		<NameTag>DatabaseName</NameTag>
		<ValueTag>DatabaseRestoreSizeMB</ValueTag>
	</GraphSource>

\end{lstlisting}

\subsubsection{Новый тип графика}

Добавить новый тип графика - \textbf{диаграмма Ганта}. Про диаграмму почитать здесь
http://ru.wikipedia.org\-/wiki/\-Диаграмма\_Ганта.

Вот исходные данные:

\begin{lstlisting}[language=XML,label=GanttChartData,caption=Данные для диаграммы Ганта]

	<GetJobsExecutionHistory>
		<JobRunStartDateTime>2013-08-14T15:20:00</JobRunStartDateTime>
		<JobRunEndDateTime>2013-08-14T15:25:00</JobRunEndDateTime>
		<JobName>JobTest1</JobName>
	</GetJobsExecutionHistory>

	<GetJobsExecutionHistory>
		<JobRunStartDateTime>2013-08-14T15:30:00</JobRunStartDateTime>
		<JobRunEndDateTime>2013-08-14T15:35:00</JobRunEndDateTime>
		<JobName>JobTest2</JobName>
	</GetJobsExecutionHistory>

	<GetJobsExecutionHistory>
		<JobRunStartDateTime>2013-08-14T15:40:00</JobRunStartDateTime>
		<JobRunEndDateTime>2013-08-14T15:45:00</JobRunEndDateTime>
		<JobName>JobTest1</JobName>
	</GetJobsExecutionHistory>

\end{lstlisting}

Вот предполагаемая декларация получения такой диаграммы:

\begin{lstlisting}[language=XML,label=GanttChartDefinition,caption=Определение для диаграммы Ганта]

	<GraphSource xsi:type="XmlFileGraphSource">
		<FileName>$INPUT$</FileName>
		<PathToItems>MSSQLResults</PathToItems>
		<ItemTag>GetJobsExecutionHistory</ItemTag>
		<DateTagBegin>JobRunStartDateTime</DateTagBegin>
		<DateTagEnd>JobRunEndDateTime</DateTagEnd>
		<NameTag>JobName</NameTag>
	</GraphSource>

\end{lstlisting}

\subsection{Программа для сбора утилиты и лицензий}

\subsubsection{Правильный выбор шаблонов}

При выборе шаблона (template), файлов запросов и файлов для построения отчётов не основываться на
лицензиях, а добавлять файлы шаблона непросредственно в утилиту. Дать возможность задать несколько
шаблонов.

\subsubsection{I18N для конфигурационной утилиты}

Для текстовых полей с названием опций сделать конфигурационный i18n файл и брать название оттуда
(сейчас имена полей зашиты в коде), так же как это сделано для основной программы. Так же задавать
файл с выбором языка. Сделать утилиту на двух языках, английский и русский с возможностью добавления
нового языка, также как и в основной программе.

\subsubsection{Новая вкладка ``Лицензии''}

Удалить подписание файла лицинзий из вкладки ``Исходные данные''. После вкладки ``Ключи'' сделать
вкладку ``Лицензии'', где можно задать файл лицензий и подписать его. Указать там файл с лицензиями
и название выходного подписанного файла лицензий. Почему? Иногда требуется подписать только новую
лизенцию и иметь отдельную вкладку удобнее. Также это позволит вынести файлы лицензий из директории
инсталлятора.

\subsubsection{Вкладка ``Сбор приложения''}

При выключенном флаге использования DNGuard делать поля с названием папки и опций недоступной для
редактирования (затенение серым цветом). Добавить опцию с названием исполняемого файла утилиты
DNGuard. Поле с опциями DNGuard сделать расширяемым при изменении диалогового окна, также как и для
поля с названием папки DNGuard, сейчас поле опций имеет фиксированную длину.

\subsubsection{32-х разрядное приложение}

При сборке дистрибутива и установке приложение устанавливается в директорию \textbf{Program
Files~(x86)} как будто приложение 32-x разрядное, хотя приложение скомпилировано как 64-х битное
приложение. Дать возможность сформировать инсталлятор для установки как 32-х, так и для 64-х
разрядных приложений.

\subsubsection{Исходные данные}

При нажатии клавиши ``...'' для поиска пути отображать не последнюю директорию, а директорию
заданную в опциях, при отсутствии такой директории или файла выводить предупреждение об ошибке с
возможностью выбора корректной директории.

\subsubsection{Директория для установки}

На машинах с установленной версией Windows 64 бит директория по умолчанию для утилиты такая,
``C:/Program Files (x86)/MSSQLServerAuditor'', хотя это не 32-x разрядное приложение и директория
должна быть ``C:/Program Files/MSSQLServerAuditor''.

\subsubsection{Список языков}

Формировать список доступных языков приложения из файла MSSQL\-ServerAuditor\-.SystemSettings\-.xml.
Языки интерфейса брать из списка available\-\_ui\_lang, языки отчётов брать из списка
available\-\_report\-\_lang. Выбранные при установке языки становятся языками по умолчанию для
interface\-\_lang и report\-\_lang в формируемом файле MSSQL\-ServerAuditor\-.SystemSettings\-.xml.

\subsubsection{Обработка лицензий}

В формирующийся дистрибутив не включать никакие лицензии, они идут отдельно.

\subsubsection{Название программы инстяллятора}

Переименовать MSSQLServerAuditorSetup в MSSQLServerAuditor. Прописывать имя компании и номер версии
программы. Для контроля просмотреть установленную программу в Control Panel.

\subsubsection{Директории дистрибутива}

Не включать директории .svn и файлы вида ``about folder'' в состав дистрибутива.

\section{Пояснения и комментарии}


