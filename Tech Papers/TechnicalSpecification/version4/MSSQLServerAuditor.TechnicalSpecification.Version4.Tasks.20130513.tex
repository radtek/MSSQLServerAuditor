\section{Изменения}

\subsection{Первого приоритета}

\subsubsection{Отображение нескольких вкладок типа HTML, XML и DataGrid}

В настоящее время возможно отображение только одной вкладки типа HTML и одной вкладки XML, причём
невозможно отключение показа какой-либо их этих вкладок. Показ же графиков очень гибкий, возможно
отображение нескольких или ни одной из графических вкладок. В дополнение к существующему
\textbf{Graph\-Preprocessor\-Dialog} ввести \textbf{HTML\-Preprocessor\-Dialog},
\textbf{XML\-Preprocessor\-Dialog} и \textbf{DataGrid\-Preprocessor\-Dialog}.

\begin{lstlisting}[language=XML,label=MSSQLServerAuditor.xml,caption=MSSQLServerAuditor.xml]

	<mssqlauditorpreprocessor preprocessor="HTMLPreprocessorDialog" name="За сегодня">

\end{lstlisting}

``DataGrid'' - обычная таблица, управляющий элемент .NET. Описание таблицы содержит описание и
название колонок, типы и шаблоны для отображения этих колонок, например шаблон ``DD.MM.YYYY'' для
полей содержащих тип даты и времени. Знание типов колонок позволит сортировать данные в таблице
правильно. Ввести возможность отображения / скрытия, изменения размера и сохранение размера и
параметров отображения / скрытия каждой колонки и типа сортировки для каждого пользователя в
отдельном конфигурационном файле. При отображении данной таблицы сортировать и изменять размер
каждой колонки основываясь на последнем предпочтении пользователя, взятом из конфигурационного
файла. Предусмотреть возможность иметь такой же конфигационный системный файл и копировать его для
новых пользователей.

\subsubsection{JavaScript для показа информации}

Для вкладок типа ``HTML'' добавить возможность использования внешних скриптов при использовании
инструкции включения внешнего скрипта. Сейчас подключение внешнего скрипта невозможно либо неясно.
Пробы с различнми путями к ``js'' скриптам к успеху не привели. Внешние файлы могут находится на
локальном диске) \textbf{javaScript} либо быть упакованными в единый файл совместно с
файлами запросов и конфигурации.

\begin{lstlisting}[language=XML,label=javascript,caption=javascript]

	<script type="text/javascript" src=".../jquery-latest.js"></script>

\end{lstlisting}

Убедиться, что возможность выполнения \textbf{javaScript} на отображаемых HTML страницах сохранится
и после обновления.

\subsubsection{Поддержка XML 2.0}

Сейчас XSL шаблоны используют XML версии \textbf{1.0}. Ввести поддержку работы с XML версией
\textbf{2.0}, чтобы в заголовке XSL документа можно определить новую версию.

\begin{lstlisting}[language=XML,label=MSSQLServerAuditor.xml,caption=MSSQLServerAuditor.xml]

	<?xml version="2.0" encoding="UTF-8"?>

\end{lstlisting}

\subsubsection{Утилита сборки дистрибутива}

Переработать существующую утилиту проверки цифровых подписей, добавить функциональность для упаковки
индивидуального дистрибутива для каждого пользователя с прописанными именами экземпляров серверов,
именами пользователей, типом авторизации этих пользователей, именем компьютера на котором запущена
программа и проверкой текущей даты (лицензия выдаётся на ограниченный срок). Реализовать как часть
комплекса программ утилит в связке с утилитой проверки зависимостей конфигурационных файлов.

\bigskip

Возможность сборки приложения как в 32, так и для 64 битной версии Windows. Сейчас доступна только
32 битная сборка.

\subsubsection{Защита конфигурации}

Брать SQL запросы не из внешних XML файлов (как это реализовано сейчас), а из одного внешнего
зашифрованного архивного файла. Для этого добавить формирование такого файла в утилиту сборки
дистрибутива. Запросы для данного архивного файла браться из соответствующих файлов с запросами,
определяемых конфигурационным файлом.

\begin{quote}

	По поводу защиты конфигурации мне видится следующее. Вообще, наверное, технически не важно,
	будет ли конфигурация лежать внутри исполняемого файла как встроенный ресурс, или же снаружи как
	архив (хотя второй вариант проще реализовать, так как не придется заниматься перелинковкой
	ресурсов при создании exe). Главное здесь - не допускать сохранения этих данных в расшифрованном
	виде на диск. То есть читать только из \textbf{Memory\-Stream}. Для этого, видимо, необходимо
	будет проводить сборку XML при создании дистрибутива для исключения использования
	\textbf{XInc\-lude}, так как есть подозрение, что вне файловой системы он не заработает.

\end{quote}

\bigskip

Сейчас, после введения конфигурационного файла с шаблонами, поддержка использования
\textbf{XInc\-lude} вообще становится не нужна, так как можно разделить файлы с запросами через
использование различных файлов с описанием в шаблоне. (Основная причина в использовании
\textbf{XInc\-lude} была в том, что был только один файл запросов, теперь их можно сделать сколько
угодно и потребность в этом функционале отпадает). Что я вижу, в принципе на начальном этапе
пересоздание \textbf{exe} файла разумная опция, но при большом количестве инсталляций думаю
превратится в проблему, поэтому опция с шифрованным архивом, содержащим файлы шаблонов и файлы
запросов, будет очень хорошим вариантом.

\bigskip

Убрать отображение открытого ключа шифрования из диалога настройки параметров. Изменение данного
ключа возможно только через конфигурирование XML файла.

\bigskip

Список языков для интерфейса и отчётов (два отдельных списка) сделать в системном конфигурационном
файле.

\bigskip

Выбор языка интерфейса и языка отчёта по умолчанию (для новых пользователей) при установке
программы.

\subsubsection{Добавление новой лицензии}

В меню ``Настройки'' добавить новый диалог для добавления новой лицензии. Пользователь должен
выбрать файл, который будет скопирован в домашнюю папку пользователя под именем
MSSQL\-Server\-Auditor.Connec\-tions.YYYMMDD.xml, что позволит ему начать работу с новым экземпляром
сервера. Возможность удаления одной или нескольких лицензий пользователем. Проверка валидности
лицензий.

\subsubsection{Кириллические и локальные переменные в исходных кодах}

При необходимости текстовые поля брать из файла локализации, убрать все текстовые значения из
исходных текстов. Просмотреть все файлы. Вот несколько примеров в файлах frmErrorLog.Designer.cs,
frmExceptionBox.Designer.cs, frmMain.Designer.cs изменить текст, брать из файла локализации:

\begin{lstlisting}[language=C,label=Labels:frmErrorLog.Designer.cs,caption=frmErrorLog.Designer.cs]

	 80 this.splitContainer2.Name = "splitContainer2";

	104 this.lbInstance.Text = "label1";

	139 this.Text = "Обнаружены ошибки";

\end{lstlisting}

\begin{lstlisting}[language=C,label=Labels:frmExceptionBox.Designer.cs,caption=frmExceptionBox.Designer.cs]

	 73 this.lblDescription.Text = "Critical error";

	 89 this.btnDetails.Name = "btnDetails";

	121 this.Text = "Critical error";

\end{lstlisting}

\begin{lstlisting}[language=C,label=Labels:frmMain.Designer.cs,caption=frmMain.Designer.cs]

	100 this.mnuOptions.Text = "Конфигурация";

	109 this.mnuHelp.Text = "Помощь";

\end{lstlisting}

\begin{lstlisting}[language=C,label=Labels:frmSettings.Designer.cs,caption=frmSettings.Designer.cs]

	131 this.lbTimeout.Text = "Таймаут соединения";

	158 this.lbTimeoutSecond.Text = "секунд";

\end{lstlisting}

Удалить неиспользуемую функциональность:

\begin{lstlisting}[language=C,label=frmMain.cs,caption=frmMain.cs]

        private void button6_Click(object sender, EventArgs e)
        {
            EstablishNewConnection();
            //NewTab();

        }

        private void mnuNewTab_Click(object sender, EventArgs e)
        {
            button6_Click(sender, e);
        }

\end{lstlisting}

\subsubsection{Шрифты в диалоговых окнах}

Использовать системные настройки, не использовать никакие шрифты, особенно шрифты указанные в
исходных текстах. В случае необходимости подготавливать файлы ресурсов.

\begin{lstlisting}[language=C,label=Fonts:frmErrorLog.Designer.cs,caption=frmErrorLog.Designer.cs]

	99 this.lbInstance.Font = new System.Drawing.Font("Tahoma", ...);

\end{lstlisting}

\subsection{Второго приоритета}

\subsubsection{Расчёт цифровых подписей}

При расчёте цифровой подписи кода SQL запросов не учитывать символы конца строки, символы табуляции
и символы пробелов. Т.е. строки запросов имеющие различие только в данных неотображаемых символах
должны иметь одинаковую цифровую подпись. Кроме изменения самой программы изменить утилиту генерации
и проверки этой подписи соответственно. Крайне желательно иметь один и тот же код для основной
программы и утилиты генерации цифровой подписи, чтобы избежать дублирования кода.

\subsubsection{Выполнение запросов в многопоточном режиме}

При выполнении запросов к группе серверов создавать отдельный поток для каждого экземпляра сервера
баз данных. Сейчас создаётся только один отдельный поток для выполнения запросов.

\subsubsection{Интерфейс}

Сохранять позицию и размер главного окна программы. При остановке и рестарте приложения
восстанавливать позицию и размер окна на момент последней остановки программы.

\bigskip

Обновление данных по нажатию \textbf{F5} при просмотре отчёта.

\bigskip

Контекстное меню, ``Открыть вложенные узлы'' или ``Закрыть вложенные узлы'' в дополнение к
контекстному меню ``Обновить узел'', ``Обновить вложенные узлы''.

\bigskip

Для отображаемого дерева слева на главном окне ввести возможность отображения иконки из XML файла,
где определяются HTML, XML, DataGrid элементы. В утилиту формирования дистрибутива добавить
возможность введения иконок как доступные ресурсы приложения. Пример отображения элементов с
иконками посмотреть в Micro\-soft SQL Mana\-gement Studio при выводе списка баз данных.

\bigskip

В диалоге подключения проверять валидность лицензии только после выбора этого подключения и выводить
уведомляющий диалог если подключение к одному или нескольким серверам невозможно с указанием причины
этого (сервер недоступен, лицензия невалидна (истёк срок действия лицензии) или иные причины).
Сейчас диалог подключения пытается проверить валидность лицензии ещё до подключения к серверу,
убрать эту функциональность, так как при большом числе серверов неэффективно.

\subsubsection{Сохранение шаблонов и результатов}

Добавить возможность сохранения всех шаблонов и результатов работы утилиты (выходных XML файлов) в
виде одного архивного файла и дать возможность отображения этого архива в утилите как будто мы
работаем с MS SQL сервером. Зачем это нужно? Идея в том чтобы переслать данный архивный файл и
открыть на другом компьютере, тем самым просмотреть всю конфигурацию удалённо, не имея доступа к
самому серверу баз данных. Почему нужно упаковывать шаблоны тоже, потому что на удалённом компьютере
может стоять другая версия утилиты с другими шаблонами. При работе утилиты в таком режиме обновление
будет недоступно (по понятным причинам) и нет необходимости подключаться к серверу баз данных. При
работе в таком режиме лицензию не проверять и не требовать.

\subsubsection{Отключение меню}

Сделать возможным отключение пользовательского меню программы. Идея в том, чтобы пользователь,
работающий на мониторе с низким разрешением, мог сосредоточился на отчётах и лишние интерфейсные
меню и строки можно было бы отключить. Сделать возможным включение меню через изменение системного
меню. Т.е. изменить системное меню Windows добавив пункт включить пользовательское меню.

\subsubsection{Проверка зависимостей конфигурационных файлов}

\begin{quote}

	Сейчас у нас появляется весьма много зависимостей между файлами конфигурации: файл подключений
	ссылается на файлы шаблонов дерева GUI, которые ссылаются на файлы запросов и сами запросы по их
	именам + на файлы преобразований XSL. Поэтому, возможно, хорошей идеей было бы организовать
	валидацию всей этой межфайловой конфигурации: проверка наличия всех шаблонов для подключений,
	проверка наличия всех используемых запросов, проверка наличия всех описанных файлов
	преобразований.

\end{quote}

\bigskip

Реализовать утилиту (в составе существующей программы проверки цифровых подписей), которая проверяет
все конфигурационные файлы перед окончательной сборкой продукта.

\bigskip

Выдавать предупреждения (это не ошибки) в следующих случаях:

\begin{itemize}

	\item SQL запрос определён не для всех поддерживаемых версий MS SQL сервера

	\item XSL файлы подготовлены не для всех поддерживаемых языков

\end{itemize}

\subsection{Третьего приоритета}

\subsubsection{Интерфейс -- вкладки}

Вкладки для подключений сделать размещающимися только над деревом списка слева, а вкладки для
графиков и текста поднять на один уровень выше, что позволит графикам и таблицам иметь больше места
на экране. Сейчас в случае одного подключения место не используется.

\subsubsection{Интерфейс -- прогресс бар}

Закрашивать отображение имени соединения в панели статуса, вместо выделения отдельной полосы слева.
Также обратить внимание что сейчас область прогресс бара отличается от других элементов в строке
статуса. Перенести кнопку остановки выполнения запроса вверх на уровень вкладки.

\subsubsection{Интерфейс -- вкладки}

Опция в диалоге конфигурации - при только одной открытой вкладке не показывать её или показывать.
При нескольких вкладках - показывать открытые вкладки всегда. Поведение похоже на браузеры
интернета, там есть такие настройки.

\subsubsection{Интерфейс -- строка обновления}

Если не было обновления страницы - показывать символы ``--'' в строке даты, чтобы статусная строка
не была плавающей. Дата всегда показывалась на одном и том же месте.

\subsubsection{Вывод графиков -- параметры}

\iffalse

Добавить конфигурационный параметр для возможности отображения графиков с либо с вертикальными либо
с горизонтальными линиями (сейчас есть возможность построения графиков только с вертикальными
линиями). При таком параметре выводить по оси \textbf{Y} время, по оси \textbf{X} - значения.

\fi

Графики типа \textbf{NameDateLine} выводить в виде линии соединяющей значения друг с другом. Сейчас
для значений выводятся пики, а не линии. Сейчас выводы графиков типа \textbf{NameDateStackedColumn}
и \textbf{NameDateLine} - очень похожи, хотя график \textbf{NameDateLine} должен отображать линии.

\bigskip

Графики типа \textbf{Column} при наличии нескольких данных отображать их друг над другом, а не
рядом. Должно быть выглядеть похоже на графики вида \textbf{NameDateStackedColumn}.

\bigskip

Добавить возможность отключения показа легенды графика пользователем. Если это легко реализуемо
сделать возможность отключения показа легенды и сделать плавающую панель легенды (в виде отдельного
окна).

\iffalse

\bigskip

После скрытия элементов графика производить перемасштабирование осей \textbf{X}, \textbf{Y}, так как
для открытых данных масштаб уже не верный.

\fi

\iffalse

\bigskip

При включенной опции отображения осей выводить завершающую вертикальную или горизонтальную линии.

\fi

\bigskip

Возможность указания единиц измерения осей графика. Параметр графика с названием оси и указанием
единиц измерения.

\bigskip

При обновлении графика не сохраняются скрытые элементы. Сохранять конфигурация скрытия / отображения
элементов при обновлении узла.

\subsubsection{Настройки программы}

При просмотре соединения в списке процессов MS SQL Server в графе ``ProgramName'' утилита должна
прописывать себя как \textbf{MSSQLServerAuditor} \textbf{0.4.0.1}, где 0.4.0.1 номер текущей версии.
Для этого изменить строку подключения к серверу (параметр тип приложения).

\bigskip

Сейчас временные файлы и настройки программы хранятся в папке ``Документы'', в моём случае это папка
``C:{\textbackslash}Docu\-ments and
Set\-tings{\textbackslash}a\_sayche\-nkoa{\textbackslash}Docu\-ments''. Думаю что более логично
поместить файлы в папку ``Приложения''. В моём случае это будет папка
``C:{\textbackslash}Docu\-ments and
Set\-tings{\textbackslash}a\_sayche\-nkoa{\textbackslash}Appli\-cation Data''. Почему? Папка
``Документы'' предназначена для документов, которые пользователь создаёт сам, а в данном случае все
файлы генерируются приложением автоматически.

\subsubsection{Конфигурационные файлы}

Изменить имя файла с локализацией интерфейса программы. Сейчас файл называется
``MSSQL\-Server\-Auditor.\-MSSQLQuery.\-i18n.\-xml''. Предлагаю изменить на
``MSSQL\-Server\-Auditor.\-i18n.\-xml''. Файл не имеет отношения к файлу с запросами как может
показаться по его имени.

\subsection{Четвёртого приоритета}

\subsubsection{Утилита проверки цифровых подписей}

При проверке файла запросов на соответствие цифровых подписей выводить имя этого запроса (сейчас
выводится только номер) и номер версии SQL сервера для этого запроса (сейчас вообще не выводится).
Причина в том что в файле запросов может быть определено несколько запросов под одним и тем же
именем для нескольких версий MS SQL сервера.

\bigskip

Добавить проверку цифровых подписей для списка баз данных. Добавить проверку цифровых подписей для
файлов с подключениями.

\bigskip

Сохранять форматирование исходного XML файла при изменении цифровых подписей. Сейчас форматирование
не сохраняется, так символы табуляции меняются на пробелы, отступы не сохраняются.

\bigskip

Не блокировать файл запросов и снимать блокировку после проверки. Сейчас даже после окончания
проверки подписей файл с запросами остаётся блокированным.

\bigskip

Сохранять настройки программы корректно. Сейчас путь к файлу с запросами не сохраняется.

\bigskip

Ввести режим проверки запросов в XML файле, т.е. перед началом проверки преобразовывать специальные
символы XML, такие как \textbf{\&lt;} в корректное представление. Рекомендую сделать дополнительный
элемент конфигурации, какой тип запроса (обычный текстовый формат или формат в виде XML).

\bigskip

Сохранять позицию и размер главного окна программы. При остановке и рестарте приложения
восстанавливать позицию и размер окна на момент последней остановки программы.

\subsubsection{Параметры подключения к серверу баз данных}

Изменить имя программы при подключении к MS SQL серверу. Сейчас имя программы \textbf{.Net SqlClient
Data Provider}, что не позволяет отличить утилиту от других .Net программ, использующих такой же
драйвер. Использовать имя и версию программы при установлении соединения с сервером баз данных,
например \textbf{MSSQLServerAuditor - 0.4.0.1}.

\subsubsection{Диалог ``О программе''}

Вывести имя локального компьютера и имя текущего пользователя Windows. Вывести тип сборки (сейчас
выводится только для ``Debug'' версии, но для продукционной выводить ``Release'').

\bigskip

Увеличить область вывода системных файлов за счёт сокращения выводов названия программы, номера
версии и копирайт информации. Выводить данные в виде таблицы (сейчас в виде простого списка) из двух
колонок: название системной компоненты и номер её версии. Первой в списке выводить номер самой
утилиты. Взять за образец диалог ``О программе'' в Micro\-soft SQL Mana\-gement Studio, там выводятся
версии системных компонент.

\subsubsection{Вывод графиков - документация}

Обновить документацию по утилите и параметрам графика.

\subsection{Пятого приоритета}

\subsubsection{MS SQL 7.x}

Исправить ошибку при работе с MS SQL версии 7.x

\subsubsection{Формирование XML файлов}

Выравнивать секции в результирующем XML файле (пробелы в начале файла заменить на символы табуляции,
выровнять расположение секций друг под другом). Сейчас:

\begin{lstlisting}[language=XML,label=existing version of xml,caption=OLD.xml]

	<Instance>
	  <Audit>
		<DateTime>
		  <GetCurrentDateTime>
		   <CurrentDateTime>18/03/2013 4:54:15 PM</CurrentDateTime>
	</GetCurrentDateTime>
	</DateTime>
		<Server>
		   <Version>Microsoft SQL Server 2012 - 11.0.2100.60 (X64)
		Feb 10 2012 19:39:15
		Copyright (c) Microsoft Corporation
		Express Edition (64-bit) on Windows NT 6.1 (X64)</Version>
	</Server>
	</Audit>
	</Instance>

\end{lstlisting}

Надо сделать:

\begin{lstlisting}[language=XML,label=new version of xml,caption=NEW.xml]

	<Instance>
		<Audit>
			<DateTime>
				<GetCurrentDateTime>
					<CurrentDateTime>18/03/2013 4:54:15 PM</CurrentDateTime>
				</GetCurrentDateTime>
			</DateTime>
			<Server>
				<Version>Microsoft SQL Server 2012 - 11.0.2100.60 (X64)
				Feb 10 2012 19:39:15
				Copyright (c) Microsoft Corporation
				Express Edition (64-bit) on Windows NT 6.1 (X64)</Version>
			</Server>
		</Audit>
	</Instance>

\end{lstlisting}

\subsubsection{Настройки файла UserSettings.xml}

Прописывать значения полей lastserver, lastlogin в пользовательский файл и при следущем октрытии
диалога подключения активировать данные поля (прокручивать список, если нужно).

\subsubsection{Настройки проекта AssemblyInfo.cs}

В поле AssemblyCompany прописать ``City24 Pty. Ltd.''. В поле AssemblyCopyright прописать
``Copyright 2013 City24 Pty. Ltd.''. В поле AssemblyConfiguration прописать тип сборки ``Debug''
или ``Release''.

\begin{lstlisting}[language=XML,label=AssemblyInfo.cs,caption=AssemblyInfo.cs]

	#if DEBUG
	[assembly: AssemblyConfiguration("Debug")]
	#else
	[assembly: AssemblyConfiguration("Release")]
	#endif

\end{lstlisting}

\subsubsection{Удаление старых и не используемых файлов}

Из папки проекта удалить все старые и не используемые файлы.

\section{Пояснения и комментарии}

\begin{quote}

	Сожалею, что возникла проблема с 7-й версией MS SQL. Починить это будет несложно. Странно
	только, что я сам нигде не нашел ссылку на DataSource, хоть и искал. Обязательно заменю
	использование свойств в запросе на вызов DataSource. Падение же программы при этом вызвано не
	проблемой в использовании провайдера, а необработанным и не запрошенным исключением после
	выполнения параллельного Task-а. То есть мне надо будет прогнать данный ошибочный сценарий. В
	целом, если это не критично, давайте отнесем на 4-й этап.

\end{quote}

Думаю что использование \textbf{DataSource} предпочтительнее прямого запроса к серверу базы данных, так как
это перекладывает задачу на драйвер, думаю что это более правильно.

\begin{quote}

	6.1.1. Тогда, видимо, есть смысл перестать притворяться XSL-файлом, а сделать отдельный
	XML-формат, частью которого (внутри <HTMLPreprocessorDialog>) и будет XSL-содержимое.

\end{quote}

Да, Вы меня поняли правильно. По историческим причинам это был XSL файл, сейчас, после расширения
его графиками он таковым быть перестал и думаю что надо сделать это правильно. Т.е. настроить HTML,
XML вывод так же как и графики с формированием заголовка и теми преимуществами что сейчас есть у
графиков.

\begin{quote}

	6.1.2. А сейчас JavaScript не работает? Нужно ли как-то реализовывать на JS взаимосвязть с
	основным приложением (переход по вкладкам, обновление запросов...)?

\end{quote}

Я не пробовал и не знаю. Я совсем не специалист в JavaScript, но хочу им воспользоваться, потому что
статических таблиц уже не хватает. Основная идея в введении JavaScript это иметь возможность
сортировать колонки в таблицах, убирать ненужные колонки, вот это хочется и решить. Переход по
вкладкам и обновление запросов я пока не планирую. Т.е. если получится прикрутить библиотеку типа
jQuery это было бы очень хорошо и решить задачи с отображениями.

\begin{quote}

	6.1.4. Насколько ``умной'' должна быть утилита? Относительно простое диалоговое приложение без
	``эффекта памяти'', или же более интеллектуальное приложение (тут мне на ум приходит HASP
	Business Studio, с которым мы работаем - основная идея - учет клиентов и выданных лицензий с
	автоматическим формированием всех нужных файлов).

\end{quote}

Да, простое приложение без ``эффекта памяти''. Пока (на данном этапе) использование подобных средств
неоправданно.

\begin{quote}

	6.1.6. Как-то при этом нужно удалять устаревшие лицензии?

\end{quote}

В принципе было бы неплохо это сделать, если можно это сделать элегантно и не так сложно.

\begin{quote}

	6.2.4. Возможно, нужно сохранять результаты в какой-то особый файл, а не просто шаблоны +
	результаты. Это необходимо потому, что нужно как-то сохранять инфо про наличие баз данных на
	инстансах.

\end{quote}

Да, я об этом не подумал, согласен.

\begin{quote}

	6.4.1. Видимо, речь идет про отдельную утилиту, которой у меня сейчас нет. Возможно, ее также
	придется переделать.

\end{quote}

Да, речь идёт об утилите. Я её Вам высылаю посмотреть её с исходными кодами. Можно переделать её,
можно написать новую и включить в неё функциональность, это решать Вам как проще сделать.


